%% abtex2-modelo-trabalho-academico.tex, v-1.9.7 laurocesar
%% Copyright 2012-2018 by abnTeX2 group at http://www.abntex.net.br/ 
%%
%% This work may be distributed and/or modified under the
%% conditions of the LaTeX Project Public License, either version 1.3
%% of this license or (at your option) any later version.
%% The latest version of this license is in
%%   http://www.latex-project.org/lppl.txt
%% and version 1.3 or later is part of all distributions of LaTeX
%% version 2005/12/01 or later.
%%
%% This work has the LPPL maintenance status `maintained'.
%% 
%% The Current Maintainer of this work is the abnTeX2 team, led
%% by Lauro César Araujo. Further information are available on 
%% http://www.abntex.net.br/
%%
%% This work consists of the files abntex2-modelo-trabalho-academico.tex,
%% abntex2-modelo-include-comandos and abntex2-modelo-references.bib
%%

% ------------------------------------------------------------------------
% ------------------------------------------------------------------------
% abnTeX2: Modelo de Trabalho Academico (tese de doutorado, dissertacao de
% mestrado e trabalhos monograficos em geral) em conformidade com 
% ABNT NBR 14724:2011: Informacao e documentacao - Trabalhos academicos -
% Apresentacao
% ------------------------------------------------------------------------
% ------------------------------------------------------------------------

\documentclass[
	% -- opções da classe memoir --
	12pt,				% tamanho da fonte
	%openright,			% capítulos começam em pág ímpar (insere página vazia caso preciso)
	oneside,			% para impressão em recto e verso. Oposto a oneside
	a4paper,			% tamanho do papel. 
	% -- opções da classe abntex2 --
	%chapter=TITLE,		% títulos de capítulos convertidos em letras maiúsculas
	%section=TITLE,		% títulos de seções convertidos em letras maiúsculas
	%subsection=TITLE,	% títulos de subseções convertidos em letras maiúsculas
	%subsubsection=TITLE,% títulos de subsubseções convertidos em letras maiúsculas
	% -- opções do pacote babel --
	english,			% idioma adicional para hifenização
	brazil				% o último idioma é o principal do documento
	]{abntex2}
% ---
% Pacotes básicos 
% ---
\usepackage{lmodern}			% Usa a fonte Latin Modern			
\usepackage[T1]{fontenc}		% Selecao de codigos de fonte.
\usepackage[utf8]{inputenc}		% Codificacao do documento (conversão automática dos acentos)
\usepackage{indentfirst}		% Indenta o primeiro parágrafo de cada seção.
\usepackage{color}				% Controle das cores
\usepackage{graphicx}			% Inclusão de gráficos
\usepackage{microtype} 			% para melhorias de justificação
% ---
		
% ---
% Pacotes adicionais, usados apenas no âmbito do Modelo Canônico do abnteX2
% ---
\usepackage{lipsum}				% para geração de dummy text
% ---

% ---
% Pacotes de citações
% ---
\usepackage[brazilian,hyperpageref]{backref}	 % Paginas com as citações na bibl
\usepackage[alf]{abntex2cite}	% Citações padrão ABNT

% --- 
% CONFIGURAÇÕES DE PACOTES
% --- 

% ---
% Configurações do pacote backref
% Usado sem a opção hyperpageref de backref
\renewcommand{\backrefpagesname}{Citado na(s) página(s):~}
% Texto padrão antes do número das páginas
\renewcommand{\backref}{}
% Define os textos da citação
\renewcommand*{\backrefalt}[4]{
	\ifcase #1 %
		Nenhuma citação no texto.%
	\or
		Citado na página #2.%
	\else
		Citado #1 vezes nas páginas #2.%
	\fi}%
% ---

% ---
% Informações de dados para CAPA e FOLHA DE ROSTO
% ---
\titulo{Título do \\ trabalho}
\autor{Daniel Cezar Salgado}
\local{Brasil}
\data{2019}
\orientador{Camila Paes Solomon}
\coorientador{Hanneli Carolina Andreazzi Tavante}
\instituicao{%
  Universidade Federal de Itajubá -- UNIFEI
  \par
  Instituto de Sistemas Elétricos e Energia -- ISEE}
\tipotrabalho{Tese (Graduação)}
% O preambulo deve conter o tipo do trabalho, o objetivo, 
% o nome da instituição e a área de concentração 
\preambulo{Monografia apresentada ao Instituto de Sistemas Elétricos e Energia, da Universidade Federal de Itajubá, como parte dos requisitos para obtenção do título de Engenheiro Eletricista.  \LaTeX.}
% ---


% ---
% Configurações de aparência do PDF final

% alterando o aspecto da cor azul
\definecolor{blue}{RGB}{41,5,195}

% informações do PDF
\makeatletter
\hypersetup{
     	%backref=true,
		pdftitle={\@title}, 
		pdfauthor={\@author},
    	pdfsubject={\imprimirpreambulo},
	    pdfcreator={LaTeX with abnTeX2},
		pdfkeywords={abnt}{latex}{abntex}{abntex2}{trabalho acadêmico}, 
		colorlinks=true,       		% false: boxed links; true: colored links
    	linkcolor=blue,          	% color of internal links
    	citecolor=black,        		% color of links to bibliography
    	filecolor=magenta,      		% color of file links
		urlcolor=blue,
		bookmarksdepth=4
}
\makeatother
% --- 

% ---
% Posiciona figuras e tabelas no topo da página quando adicionadas sozinhas
% em um página em branco. Ver https://github.com/abntex/abntex2/issues/170
\makeatletter
\setlength{\@fptop}{5pt} % Set distance from top of page to first float
\makeatother
% ---

% ---
% Possibilita criação de Quadros e Lista de quadros.
% Ver https://github.com/abntex/abntex2/issues/176
%
\newcommand{\quadroname}{Quadro}
\newcommand{\listofquadrosname}{Lista de quadros}

\newfloat[chapter]{quadro}{loq}{\quadroname}
\newlistof{listofquadros}{loq}{\listofquadrosname}
\newlistentry{quadro}{loq}{0}

% configurações para atender às regras da ABNT
\setfloatadjustment{quadro}{\centering}
\counterwithout{quadro}{chapter}
\renewcommand{\cftquadroname}{\quadroname\space} 
\renewcommand*{\cftquadroaftersnum}{\hfill--\hfill}

\setfloatlocations{quadro}{hbtp} % Ver https://github.com/abntex/abntex2/issues/176
% ---

% --- 
% Espaçamentos entre linhas e parágrafos 
% --- 

% O tamanho do parágrafo é dado por:
\setlength{\parindent}{1.3cm}

% Controle do espaçamento entre um parágrafo e outro:
\setlength{\parskip}{0.2cm}  % tente também \onelineskip

% ---
% compila o indice
% ---
\makeindex
% ---

% ----
% Início do documento
% ----
\begin{document}

% Seleciona o idioma do documento (conforme pacotes do babel)
%\selectlanguage{english}
\selectlanguage{brazil}

% Retira espaço extra obsoleto entre as frases.
\frenchspacing 

% ----------------------------------------------------------
% ELEMENTOS PRÉ-TEXTUAIS
% ----------------------------------------------------------
% \pretextual

% ---
% Capa
% ---
\imprimircapa
% ---

% ---
% Folha de rosto
% (o * indica que haverá a ficha bibliográfica)
% ---
%\imprimirfolhaderosto*
% ---

% ---
% Inserir folha de aprovação
% ---

% Isto é um exemplo de Folha de aprovação, elemento obrigatório da NBR
% 14724/2011 (seção 4.2.1.3). Você pode utilizar este modelo até a aprovação
% do trabalho. Após isso, substitua todo o conteúdo deste arquivo por uma
% imagem da página assinada pela banca com o comando abaixo:
%
% \begin{folhadeaprovacao}
% \includepdf{folhadeaprovacao_final.pdf}
% \end{folhadeaprovacao}
%
\begin{folhadeaprovacao}

  \begin{center}
    {\ABNTEXchapterfont\large\imprimirautor}

    \vspace*{\fill}\vspace*{\fill}
    \begin{center}
      \ABNTEXchapterfont\bfseries\Large\imprimirtitulo
    \end{center}
    \vspace*{\fill}
    
    \hspace{.45\textwidth}
    \begin{minipage}{.5\textwidth}
        \imprimirpreambulo
    \end{minipage}%
    \vspace*{\fill}
   \end{center}

   \assinatura{\textbf{\imprimirorientador} \\ Orientador} 
   \assinatura{\textbf{\imprimircoorientador} \\ Coorientador}
   \assinatura{\textbf{Professor} \\ Convidado 2}
   %\assinatura{\textbf{Professor} \\ Convidado 3}
   %\assinatura{\textbf{Professor} \\ Convidado 4}
      
   \begin{center}
    \vspace*{0.5cm}
    {\large\imprimirlocal}
    \par
    {\large\imprimirdata}
    \vspace*{1cm}
  \end{center}
  
\end{folhadeaprovacao}
% ---

% ---
% Agradecimentos
% ---
\begin{agradecimentos}
	
	
\end{agradecimentos}
% ---

% ---
% Dedicatória
% ---
\begin{dedicatoria}
   \vspace*{\fill}
   \centering
   \noindent
   \textit{ Dedicatória vem aqui :p}  \vspace* {\fill}
\end{dedicatoria}
% ---

% ---
% RESUMOS
% ---

% resumo em português
\setlength{\absparsep}{18pt} % ajusta o espaçamento dos parágrafos do resumo
\begin{resumo}
 Resumo :p

 \textbf{Palavras-chave}: 
\end{resumo}

% resumo em inglês
\begin{resumo}[Abstract]
 \begin{otherlanguage*}{english}
   This is the english abstract.

   \vspace{\onelineskip}
 
   \noindent 
   \textbf{Keywords}: latex. abntex. text editoration.
 \end{otherlanguage*}
\end{resumo}


% ---
% inserir lista de ilustrações
% ---
\pdfbookmark[0]{\listfigurename}{lof}
\listoffigures*
\cleardoublepage
% ---

% ---
% inserir lista de quadros
% ---
\pdfbookmark[0]{\listofquadrosname}{loq}
\listofquadros*
\cleardoublepage
% ---

% ---
% inserir lista de tabelas
% ---
\pdfbookmark[0]{\listtablename}{lot}
\listoftables*
\cleardoublepage
% ---

% ---
% inserir lista de abreviaturas e siglas
% ---
\begin{siglas}
  \item[LSTM] \textit{Long-Short Term Memory}
  \item[AI] \textit{Artificial Intelligence}
\end{siglas}
% ---

% ---
% inserir lista de símbolos
% ---
\begin{simbolos}
  \item[$ \Gamma $] Letra grega Gama
%  \item[$ \Lambda $] Lambda
%  \item[$ \zeta $] Letra grega minúscula zeta
%  \item[$ \in $] Pertence
\end{simbolos}
% ---

% ---
% inserir o sumario
% ---
\pdfbookmark[0]{\contentsname}{toc}
\tableofcontents*
\clearpage
%\cleardoublepage
% ---



% ----------------------------------------------------------
% ELEMENTOS TEXTUAIS
% ----------------------------------------------------------
\textual

% ----------------------------------------------------------
% Introdução (exemplo de capítulo sem numeração, mas presente no Sumário)
% ----------------------------------------------------------
\chapter{Introdução}
% ----------------------------------------------------------

%\begin{itemize}
%	\item Problema!! 
%	\item Predição de cargas
%	\item   de curto prazo
%	\item Qual a necessidade?
%	\item Como é feito atualmente
%	\item Onde isso falha?
%	\item 
%	\item Como ML pode ajudar na predição de cargas
%	\item  Natureza dos dados - sequenciais
%\end{itemize}

	Redes neurais artificiais são capazes de aprender novas relações não lineares e complexas em series temporais que com técnicas de regressão não seria possível. Ela identifica correlações entre diversos parâmetros de entrada.  




	
	
	% Referência ?? --------------------
	Sistemas de armazenamento de energia elétrica ainda são tecnologias de elevado custo e portanto não são largamente utilizadas para suprir grandes blocos de energia. Assim, deve sempre haver um balanço entre a energia gerada e consumida, incluindo perdas, ao longo do sistema, que quando desproporcionais pode levar a alteração dos valores de tensão nas barras do sistema, retirando-os da faixa aceitável, ou deixar o sistema mais susceptíveis a pequenas falhas em equipamento e linhas. 
	
	O sistema elétrico é uma rede complexa e interligada em diferentes pontos com barras responsáveis pela geração de energia, linhas de transmissão por levar a energia gerada até as barras de consumidores, cujas cargas utilizarão essa energia. Previsão do comportamento e tendências do sistema ao longo do tempo auxilia na tomada de decisão quanto ao planejamento e operação do sistema. É importante, então, uma estimativa satisfatória da carga a ser consumida, para que possa determinar a quantidade de energia a ser gerada e quais os usinas geradoras que podem suprir essa necessidade conhecendo o sistema físico e possuir modelos adequados para representá-lo sabendo a capacidade e limites de cada componente que compõe o sistema.
	
	% Referências ----------------------------------------
	Energia elétrica é um recurso de grande importância para a sociedade e economia. Todas as pessoas dependem do seu correto funcionamento, com valores aceitáveis e disponibilidade segura e confiável. Ela contêm diversos equipamento com vida útil limitada e necessidade de manutenções preventivas, e também recursos naturais limitantes na capacidade de geração, tais como volume de reservatórios de usinas hidroelétricas ou reservas de carvão e gás para usinas termoelétricas. 
	
	Gerir de forma otimizada os recursos existentes é uma tarefa que pode determinar o sucesso ou fracasso de um empreendimento. Assim como determina o custo do empreendimento. Conhecer a demanda dos recursos pode ajudar no planejamento financeiro de acordo com as necessidades, e com o retorno sobre o invertimento desejado, podendo utilizar da geração e transmissão de energia mais caras ou mais mais baratas. 
	
	Para as concessionárias de distribuição, é interessante ter a previsão de cargas da sua concessão para o planejamento das transações de compra da energia que ela revenderá. Podem otimizar os fornecedores, seus preços e a quantidade de energia necessária, minimizando seus gastos em diferentes horários do dia (e com diferentes tarifas).\cite{ReviewSmartMeterLoadForecastingTechniques}
	
	
	 
	
	
	Gerenciamento de operações, quantidade de geração e consumo e transmissão.
Consumidores podem otimizar a utilização de seus equipamentos elétricos em diferentes horários para diferentes tarifas.  



\subsection{Horizontes de previsão}
	Horizontes de previsão são o período de tempo para cada previsão. Elas podem ser de longo, médio, curto ou curtíssimo prazo, cada uma com diferentes utilidades para o planejamento, gerenciamento e operação de cada parte do sistema.	Podem ainda ter diferentes escalas, podem ser desde o consumidor final até todo o sistema elétrico. \cite{ReviewSmartMeterLoadForecastingTechniques} E ainda para cada horizonte e escala, os dados apresentam particularidades que permitem a utilização de diferentes técnicas e modelos matemáticos para a previsão. 
	
	Operação do sistema - confiabilidade em níveis aceitáveis mas com custos mínimos  - diferentes horizontes

\subsubsection{Previsão de longo prazo}
	O planejamento e desenvolvimento de cada parte do sistema, em especial do sistema de transmissão de energia elétrica, requer simulações e previsões de carga e geração em diferentes senários. Envolvem inúmeras fontes de dados de diferentes domínios, tais que todas são geradas por outras previsões, dessa forma as incertezas e riscos das medidas e modelos se propagam ao longo de cada etapa e tendem a aumentar. 
	
	Essa previsão com horizonte de décadas a frente é utilizada para expansão e crescimento do sistema, maximizando lucros, segurança e confiabilidade. Modificações do sistema elétrico envolvem elevados custos, e necessitam de tempo longo para que sejam implementadas, tais como construção de novas usinas, subestações e linhas de transmissão, licenças ambientais, compras de equipamentos, entre outros, e requerem planejamentos de obras civis e financeiro para serem realizadas.  
	
	Dessa forma, é mais importante saber o que será necessário do que quando, então data e horário não é tão importante quanto a localização e capacidade. 	
	
	% Legislações e políticas de utilização também são realizadas nesse horizonte.
	
	
\subsubsection{Previsão de médio prazo}
	
	O sistema elétrico é uma rede de componentes que deve funcionar intermitentemente, sem a possibilidade de parada de fornecimento de energia para o consumidor e para isso rotas e conexões alternativas são possíveis em um sistema malhado. Mas concomitantemente, equipamentos necessitam de paradas temporárias para manutenção, uma vez que possuem grande valor financeiro e requerem cuidados adequados para seu correto funcionamento e manterem sua vida útil prolongada e manter a prevenção de falhas e possíveis interrupções no fornecimento. Mas isso só é possível com um calendário adequado de paradas e pode-se criar esse calendário com base na previsão de cargas que o sistema precisará suprir e ainda minimizar os custos de manutenção, em horários e com recursos necessários. 
	
	Previsão de médio prazo está relacionado com o planejamento de operações do sistema elétrico, criando diferentes cenários de operação com diferentes estratégias de operação e contenção para cada. Melhorias na infraestrutura e novos componentes e conexões no sistema é possível em menor escala. Com horizonte de alguns meses até o dia anterior da operação, as previsões auxiliam na tomada de decisão para melhor utilização de equipamentos e estrutura do sistema, otimizando sua performance e maximizando o retorno sobre o investimento feito sobre eles. 
	
	
\subsubsection{Previsão de curto prazo}
	


	Melhorias e trocas nos sistemas de proteção
	
	Reparos e condicionamento de equipamentos de monitoramento e proteção
	
	Planejamento até a hora anterior
	
	Ações preventivas, cancelamento de manutenção, contenção de faltas, redispacho de energia
	
	Indisponibilidade de energia - falta e falha no sistema
	
	Análise de contingência - mudança inesperada da condição de carga, condição climática, ou falhas	
	
	Minimizar riscos na operação
	
\subsubsection{Previsão de curtíssimo prazo}
	


\subsection{ANNSTLF}
	ANNSTLF (Artificial Neural Network Short-Term Load Forecaster) é uma rede neural artificia para previsão de carga horária do sistema elétrico criado pela Southern Methodist University and PRT, Inc. com patrocínio da  Electric Power Research Institute (EPRI) nos Estados Unidos. Ele permite identificar correlações entre parâmetros de entrada como carga elétrica, condições climáticas (como temperatura e umidade relativa) , dia da semana e do ano, horário do dia em séries temporais. É também utilizado um parâmetro para feriados e dias especiais, tal que sabe-se que a carga pode variar da série histórica em tais dias e como eles aparecem com pouca frequência, uma vez por ano, a rede não consegue se adaptar e os erros se tornam elevados. \cite{khotanzad1998annstlf}

O programa desenvolvido necessita de 2 a 3 anos de dados históricos de todas as variáveis de entrada para seu treinamento, e com ele é possível predizer a condição de carga horária para os 35 dias seguintes, sendo atualizada com o passar do tempo com o fornecimento dos dados a cada hora.

A arquitetura da rede neural implementada é baseada em Feedforward Perceptron Multicamada (MLP, do inglês, Multi-layer Perceptron) e foi escolhida por ser de grande complexidade e por outras redes como RNN não terem apresentado maiores vantagens em seus testes. Seu treinamento utiliza o conjunto de treinamento cruzado para evitar sobretreinamento (overtraining). 


% Como é feito o treinamento: -----------------------------------------------------------------------
% To avoid over-training, the cross-validation method is used. The training set is divided into two sets. For instance, if three years of data is available, it is divided into a two-year and a one-year set. The first set is used to train the MLP and the second set is used to test the trained model after every few hundred passes over the training data. The error on the validation set is examined. Typically this error decreases as the number of passes over the training set is increased until the ANN is over-trained, as signified by a rise in this error. Therefore, the training is stopped when the error on the validation set starts to increase. This procedure yields the appropriate number of epochs over the training set. The entire three years of data is then used to re-train the MLP using this number of epochs.
% ---------------------------------------------------------------------------------------------------

\chapter{Revisão da literatura}

\section{Tecnologias Utilizadas}

\subsection{Python}

% Python ------------------------------------------------------
Python é uma linguagem de programação interpretada, de alto nível, funcional, orientada a objetos e com semântica dinâmica criada na década de 1990 e com sua sintaxe semelhante a um pseudocódigo.Sua sintaxe se assemelha a linguagem natural humana, o que permite rápido aprendizado e desenvolvimento, assim como fácil leitura, o que permite também fácil manutenção. Tais características fazem com que a linguagem seja muito utilizada para prototipagem, tal que é possível escrever um programa complexo, com grande facilidade, menos linhas de código e mais rápido que em outras como Java ou C++. Por outro lado ela não possui a mesma velocidade de execução como essas. Linguagens funcionais e orientadas a objetos permitem a divisão do código em pequenas partes, tais que possam ser largamente reutilizadas.
 
Como suporta módulos e pacotes de extensão, novos pacotes podem ser criados para estender as suas funcionalidades. Por esse motivo, tem sido utilizada no meio científico, com pacotes específicos voltados a estrutura de dados, visualização, computação científica, entre outros. Por fim, python possui licença livre e de grande portabilidade, podendo ser utilizada em qualquer tipo de sistema. 

\subsection{Jupyter Notebook}
Jupyter é uma aplicação de internet interativa de um ambiente de programação com licença livre e código aberto. Um notebook é uma documento que combina execução de código, linguagem rica, fórmulas matemáticas, gráficos e media tal que facilita a leitura humana. O aplicativo permite sua execução em um navegador de internet, podendo ser local ou em um servidor remoto. 

\subsection{Google Colaboratory - Colab}
Google Colaboratory, Colab, é uma implementação do aplicativo Jupyter Notebook que é executado na nuvem utilizando recursos computacionais fornecidos pela Google, tais como memória RAM, CPU e GPU, os quais são de grande interesse uma vez que projetos com grande quantidade de dados normalmente consomem muito deles.

%  Faltam referências ---------------------------
Redes neurais tem uma grande necessidade de processamento paralelo entre os nós de cada camada, logo, para diminuir o tempo de treinamento, pode-se utilizar uma central de processamento com múltiplos núcleos que permita processamento paralelo. CPUs podem possuir múltiplos núcleos, mas reduzidos a algumas dezenas, já GPUs possuem microprocessadores capazes de operar em paralelo para realizar um elevado número de operações mais simples em um tempo muito mais baixo. O uso de GPUs é recomendado para a utilização de redes neurais artificiais, podendo realizar as muitas operações em paralelo, aumentando a performance. 
 
Faz parte da plataforma de aplicativos de internet fornecidos pela empresa e permite que o \textit{notebook} seja aberto e compartilhado pelo navegador de internet e ainda possui as bibliotecas mais utilizadas para desenvolvimento de aprendizado de máquina em python já instaladas e prontas para serem importadas. 

\subsection{Tensorflow}
Tensorflow é uma biblioteca gratuita de código aberto criada e mantida pela Google, escrita em C++ e com suporte para diversas outras, tal como python, e criada para desenvolvimento de modelos de inteligência artificial e aprendizado de máquina. Utiliza de tensores (vetores multidimensionais) como arestas de dados, tais quais fluem entre nós, que representam operações. Então uma aresta carrega informação de um nó para outro, e o resultado da operação se torna a entrada para outra. Um benefício desse tipo de programação é que as operações não precisam mais ser realizadas sequencialmente, mas podem ser paralelas. Dessa forma pode-se agendar as tarefas no processador de maneiras mais eficientes, distribuindo  melhor entre os recursos, como várias unidades de processamento em uma única máquina ou até mesmo em diferentes máquinas. 

\subsection{Keras}
Keras é uma Interface de Programação de Aplicação (API, do inglês \textit{Application Programming Inferface}) de alto nível para desenvolvimento de redes neurais e escrita em python. Ela roda sobre outras bibliotecas de aprendizado de máquina, como Tensorflow, utilizando seus recursos como \textit{backend}. Foi criada com o objetivo de facilitar o desenvolvimento de modelos de rede neurais para experimentação, pesquisa e até produção, e para isso já possui implementado os modelos mais utilizados, como LSTM. Dentre seus benefícios estão a rápida prototipagem, sendo modulares e extensivas, suporte para redes neurais convolucionais e recorrentes e suporte a CPU e GPU.


% Falta Acabar ----------------------------------------
Tais ferramentas compõem arsenal no qual podem ser utilizadas em conjunto mantendo uma boa performance. Keras mantém a interface de fácil sintaxe em python para fácil criação de modelos de redes neurais com Tensorflow, mas não herda seu gargalo de baixa performance, uma vez que esse possui núcleo de alto desempenho escrito em C++. E ainda utilizando Google Colab e os seus recursos de processamento remoto com GPU, aumentam ainda mais a velocidade de processamento e treinamento das redes neurais testadas. 


\chapter{Modelagem teórica}

\section{Base de dados}

	Em 2007, a distribuidora de energia elétrica da Irlanda ESB Networks lançou um projeto piloto de tecnologia de medidores inteligentes de energia elétrica para melhorar seu entendimento das necessidades para o desenvolvimento de medidores inteligentes \cite{CER}. Com esse projeto, a empresa tinha como objetivo validar seus projetos de infraestrutura necessária, tais como comunicação, hardware e softwares; quantificar possíveis problemas que possam surgir com a tecnologia; e adquirir conhecimento de negócios para um projeto de tal dimensão, tais como novos modelos de tarifas e comportamento no consumo dos seus consumidores. Durante os testes, foram realizadas medições em 6700 residências e pequenas e médias empresas, tais que fossem representativas do país, a cada 30 minutos por um período de 18 meses entre julho de 2009 e dezembro de 2010. Apos as medições e relatórios, os dados foram disponibilizados para fins de testes e pesquisas, e utilizado nesse trabalho.	
	
	Modelos preditivos, os quais realizam previsões baseados em valores passados, dependem de dados sequenciais, isso é, dados em que as observações são ordenadas. Tais dados podem ser temporais ou não. Como não temporal pode-se citar dados para processamento de linguagem natural, uma vez que a sequência das palavras na frase podem mudar completamente o significado dela mas independe diretamente da data quando ela foi dita. E como temporais pode-se citar os valores de ações no mercado financeiro e consumo de energia elétrica em um determinado período.
	
	A base de dados original é composta por três colunas. A identificação do medidor, a energia consumida durante o intervalo de 30 minutos de medição (em kWh) e um código de 5 dígitos representante das datas. Esse código era composto pelos 3 primeiros dígitos representando o dia do ano que a medida foi realizada (dia 1 é 1 de janeiro de 2009), e os 2 últimos dígitos o horário do dia (1 a 48 para cada 30 minutos, com 1 sendo 00:00:00 até 00:29:59). Cada um dos horários foram transformados em timestamps sequenciais com horário e dia separadamente. 
	
		
	
%	Sabe-se LSTM apresenta melhor resultado com séries temporais que apresentam característica estatísticas, tais como média e variância, constantes ao longo da série. Dessa forma, foram realizadas duas transformações nos dados já limpos: normalização e 

\chapter{Análise experimental}

\section{Análise 1}

\lipsum[24]


\chapter{Resultados e discussões}

\section{Resultado e discussão 1}

\lipsum[25]

% ----------------------------------------------------------
% Finaliza a parte no bookmark do PDF
% para que se inicie o bookmark na raiz
% e adiciona espaço de parte no Sumário
% ----------------------------------------------------------
\phantompart

% ---
% Conclusão
% ---
\chapter{Conclusão}
% ---

\lipsum[31-33]

% ----------------------------------------------------------
% ELEMENTOS PÓS-TEXTUAIS
% ----------------------------------------------------------
\postextual
% ----------------------------------------------------------

% ----------------------------------------------------------
% Referências bibliográficas
% ----------------------------------------------------------
\bibliography{abntex2-modelo-references}

% ----------------------------------------------------------
% Glossário
% ----------------------------------------------------------
%
% Consulte o manual da classe abntex2 para orientações sobre o glossário.
%
%\glossary

% ----------------------------------------------------------
% Apêndices
% ----------------------------------------------------------

% ---
% Inicia os apêndices
% ---
\begin{apendicesenv}

% Imprime uma página indicando o início dos apêndices
\partapendices

% ----------------------------------------------------------
\chapter{Quisque libero justo}
% ----------------------------------------------------------

\lipsum[50]

% ----------------------------------------------------------
\chapter{Nullam elementum urna vel imperdiet sodales elit ipsum pharetra ligula
ac pretium ante justo a nulla curabitur tristique arcu eu metus}
% ----------------------------------------------------------
\lipsum[55-57]

\end{apendicesenv}
% ---


% ----------------------------------------------------------
% Anexos
% ----------------------------------------------------------

% ---
% Inicia os anexos
% ---
\begin{anexosenv}

% Imprime uma página indicando o início dos anexos
\partanexos

% ---
\chapter{Morbi ultrices rutrum lorem.}
% ---
\lipsum[30]

% ---
\chapter{Cras non urna sed feugiat cum sociis natoque penatibus et magnis dis
parturient montes nascetur ridiculus mus}
% ---

\lipsum[31]

% ---
\chapter{Fusce facilisis lacinia dui}
% ---

\lipsum[32]

\end{anexosenv}

%---------------------------------------------------------------------
% INDICE REMISSIVO
%---------------------------------------------------------------------
\phantompart
\printindex
%---------------------------------------------------------------------

\end{document}
